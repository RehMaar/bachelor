\section{Реализация Class-Based WFQ в ядре Linux}

	\subsection{Описание API планировщика и классификатора пакетов}

	% Информация из комментариев к ним.
	% Файлы: linux/net/sched/sch_api.c, linux/include/net/{sch_generic.h,pkt_sched.h}. 
	
        API ядра Linux для реализации пакетных планировщиков состоит из двух частей:
		\begin{itemize}
            \item интерфейс управления очередями;
            \item интерфейс классов трафика.
		\end{itemize}


        В общем, дисциплина обслуживания -- это чёрный ящик, который может ставить пакеты в очереди и вынимать их из очереди, когда устройство готово к отправке, в порядка и во время, определёнными спрятанным в ящике алгоритмом.

        ДО делятся на две категории:
		\begin{itemize}
            \item очереди, которые не имеют внутренней структуры, видимой снаружи;
            \item планировщики, которые разделяют все пакеты на классы трафика, используя пакетные классификаторы.
		\end{itemize}

        In turn, классы могут иметь подклассы (как правило, очереди). 



	Таким образом, в качестве описания алгоритма дисциплины обслуживания служат описания
	алгоритмов всех функций, которые требует API ядра.

	\subsection{Алгоритм CBWFQ}
	


