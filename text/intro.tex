\phantomsection
\section*{ВВЕДЕНИЕ}
\addcontentsline{toc}{section}{ВВЕДЕНИЕ}

В наши дни на фоне высокого спроса на высокоскоростную и надёжную передачу данных
остро встаёт проблема обработки трафика и обеспечения должного уровня обслуживания
в узлах компьютерных сетей. Крупные производители сетевого оборудования
предоставляют эффективные решения обозначенных проблем, однако оборудование и программное
обеспечение стоят немалых денег, что делает их недоступными для массового пользователя.
Большое распространение в качестве сетевого ПО получили дистрибутивы Linux,
которые из-за многолетнего использования имеют богатый набор возможностей в этой области.
Таким образом, встаёт вопрос о интеграции возможностей,
предоставляемых гигантами индустрии, с широко распространённым открытым вариантом.

В связи с этим было решено изучить дисциплину обслуживания, разработанную компанией Cisco,
известным вендором на рынке сетевых решений, и внедрить её в ядро последней стабильной версии Linux (4.15.3). 
В качестве дисциплины обслуживания был выбран взвешенный алгоритм честного
обслуживания основанного на классах (Class-Based Weighted Fair Queuing, CBWFQ).

CBWFQ является расширением функциональности широко известного взвешенного алгоритма
честного обслуживания очереди (Weighted Fair Queuing). Он поддерживает
определение пользовательских классов трафика на основе ряда критериев соответствия
(протокол, входящий интерфейс и т.д.) и назначение их характеристик, которые
отвечают за выделяемые классу ресурсы (вес, пропускная способность, максимальное
количество пакетов в очереди, задержка). Такой подход предоставляет гибкую настройку
распределения пропускной способности канала между классами трафика и оказывается весьма
эффективным в передаче данных в сравнении с рядом других популярных дисциплин.


\textbf{Цель работы} -- реализовать алгоритм Class-Based WFQ в виде модуля ядра Linux, основываясь на имеющихся
описаниях в соответствующей литературе. 
Для выполнения цели работы необходимо выполнить следующие задачи:
\begin{enumerate}
    \item Проанализировать и сравнить дисциплины обслуживания PQ, CBQ, HTB, HFSC, FWFQ, CBWFQ.
    \item Восстановить алгоритмы Class-Based WFQ.
    \item Настроить среду для реализации и тестирования.
    \item Реализовать модуль ядра CBWFQ в ядре Linux.
    \item Реализовать интерфейс утилиты tc для управления модулем.
    \item Провести тестирование.
\end{enumerate}

Сравнительный анализ дисциплин обслуживания в работе основываются на научных статьях, документации и исходных кодах;
архитектура подсистемы Linux по управлению качеством обслуживания затрагивается в книге \cite{tcpip}, однако
в силу отсутствия полной документации все выводы о структуре делались на основе исходного кода ядра Linux.
Алгоритм CBWFQ воссоздавался на основе документации Cisco, книг \cite{tcpip} и \cite{Vagesna} и исходного кода
существующих дисциплин обслуживания. 
