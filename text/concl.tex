\phantomsection
\section*{ЗАКЛЮЧЕНИЕ}
\addcontentsline{toc}{section}{ЗАКЛЮЧЕНИЕ}


	В результате выполнения работы были достигнуты следующие цели:
	\begin{itemize}
		\item проведён сравнительный анализ дисциплин обслуживания PQ, CBQ, HTB, HFSC, FWFQ, CBWFQ;
			  дано краткое описание алгоритмов и приведены их слабые и сильные стороны;
		\item проведено исследование архитектуры подсистемы контроля качества обслуживания ядра Linux,
			  описаны механизмы работы подсистемы;
		\item реализован и протестирован модуль дисциплины обслуживания Class-Based WFQ для ядра Linux;
		\item реализован модуль для утилиты tc для взаимодействия с модулем дисциплины обслуживания. 
	\end{itemize}

	В дальнейшем работу можно развить в следующих направлениях.
	\begin{enumerate}
		\item Реализация взвешеннего алгоритм раннего обнаружения (WRED) для возможности
			  конфигурации политики отбрасывания для модуля CBWFQ;
		\item Доработка работы до дисциплины Low-Latency Queuing (LLQ), которая
			  совершенствует дисциплину CBWFQ с помощью добавления приоритетных очередей,
			  использующиеся для чувствительного к задержкам трафика.

	\end{enumerate}
