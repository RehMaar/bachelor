% Сравнительный анализ алгоритмов qdisc
	\subsection{Дисциплины обслуживания очередей}

		%Понятие ДО. Зачем они и где они.обработки
		\textbf{Дисциплина обслуживания очередй (queuing discipline, qdisc)} -- это алгоритм выбора и обработки пакета
		из очереди пакетов, где под очередью подразумевается буфер пакетов. \\
		
%		ДО делятся на \textbf{приоритетные} и \textbf{бесприоритетные} и \textbf{классовые} и \textbf{бесклассовые}.
		ДО бывают следующих видов.
		\begin{description}
			\item[Приоритетные] -- ДО, которые назначают для разных видов трафика 
			\item[Бесприоритетные] -- ДО, которые не приоретизируют трафик. К таким ДО относится, к примеру, fifo.
			\item[Классовые]
			\item[Бесклассовые]
		\end{description}


	\subsection{Алгоритмы, реализованные в ядре Linux}

			\subsection{Бесклассовые алгоритмы (Classless)}
				\begin{itemize}
					\item \textbf{CODEL} (controlled-delay active queue management)
					\item \textbf{FIFO} (first in first out)
					\item \textbf{FQ} (fair queue)
					\item \textbf{RED} (random earky detection)
					\item \textbf{SFB} (stochastic fair blue)
				\end{itemize}

			\subsection{Классовые алгоритмы (Classful)}
				\begin{itemize}
               		\item CBQ (class based queueing) -- обработка пакетов в очередях 
               		\item HFSC (heirarchy fair service curve's control) --
               		\item HTB  (heirarchy token bucket) -- 
				\end{itemize}
			

%
%	\subsection{Методы управления трафиком в Linux}
%
%		Под управлением трафиком подразумевают набор механизов и операций, с помощью которого пакеты организовывают в очередь
%		для приёма/передачи через сетевой интерфейс; термин включает в себя механизмы принятия решений, какие пакеты принимать и
%		с какой скоростью на входящем интерфейсе, определения пакетов для передачи, их порядка и скорость передачи на исходящем устройстве.[1]
%
%		Рассказать, как решают проблемы с узкими местами
%		в маршруте трафика.



%
%	\subsection{Обзор существующих ДО в ядре Linux}
%
%		Какие есть, где используются, почему нужно ещё.
%
%        CLASSLESS
%        	* codel -- controlled-delay activa queue management
%        	* fifo	-- 
%        	* fq	-- fair queue
%        	* red   -- random earky detection 
%        	* sfb	-- stochastic fair blue 
%
%        CLASSFUL
%        	* cbq	-- class based queueing 
%        	* hfsc	-- heirarchy fair service curve's control under linux
%        	* htb	-- heirarchy token bucket
%
%	\subsection{Обоснование выбора}
%
%		Описать идею алгоритма CBWFQ
%
%		CBWFQ -- комбинация CBQ и WFQ
%
%		Описать приятности CBWFQ.
%
%	\subsection{Выводы}
%
%		Так как я не могу найти непосредственное описнание
%		алгоритма, придумаю его на основании всего того, что слышу.
%
%
%	\subsection{Список использованной литературы}
%		\begin{enumerate}
%			\item http://computerlib.narod.ru/html/traffic.htm, дата обращения 10-12-2017
%		\end{enumerate}
