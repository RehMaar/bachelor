\documentclass[14pt] {ncc}
\usepackage[utf8] {inputenc}
\usepackage[T2A]{fontenc}
\usepackage[english, russian] {babel}
\usepackage[usenames,dvipsnames]{xcolor}
\usepackage{listings,longtable,amsmath,amsfonts,graphicx}
\usepackage{indentfirst}
\usepackage{bytefield}
\usepackage{multirow}
\usepackage{float}
\usepackage{caption}
\usepackage{subcaption}
\captionsetup{compatibility=false}
\usepackage{tabularx}

\linespread{1.3}


\usepackage[left=2cm,right=2cm,top=2cm,bottom=2cm,bindingoffset=0cm]{geometry}

\begin{document}

% Введение
%	\phantomsection
\section*{ВВЕДЕНИЕ}
\addcontentsline{toc}{section}{ВВЕДЕНИЕ}

В наши дни на фоне высокого спроса на высокоскоростную и надёжную передачу данных
остро встаёт проблема обработки трафика и обеспечения должного уровня обслуживания
в узлах компьютерных сетей. Крупные производители сетевого оборудования
предоставляют эффективные решения обозначенных проблем, однако оборудование и программное
обеспечение стоят немалых денег, что делает их недоступными для массового пользователя.
Большое распространение в качестве сетевого ПО получили дистрибутивы Linux,
которые из-за многолетнего использования имеют богатый набор возможностей в этой области.
Таким образом, встаёт вопрос о интеграции возможностей,
предоставляемых гигантами индустрии, с широко распространённым открытым вариантом.

В связи с этим было решено изучить дисциплину обслуживания, разработанную компанией Cisco,
известным вендором на рынке сетевых решений, и внедрить её в ядро последней стабильной версии Linux. 
В качестве дисциплины обслуживания был выбран взвешенный алгоритм честного
обслуживания основанного на классах (Class-Based Weighted Fair Queuing, CBWFQ).

CBWFQ является расширением функциональности широко известного взвешенного алгоритма
честного обслуживания очереди (Weighted Fair Queuing). Он поддерживает
определение пользовательских классов трафика на основе ряда критериев соответствия
(протокол, входящий интерфейс и т.д.) и назначение их характеристик, которые
отвечают за выделяемые классу ресурсы (вес, пропускная способность, максимальное
количество пакетов в очереди, задержка). Такой подход предоставляет гибкую настройку
распределения пропускной способности канала между классами трафика и оказывается весьма
эффективным в передаче данных в сравнении с рядом других популярных дисциплин.\cite{thesis}

\textbf{Цель работы} -- реализовать алгоритм Class-Based WFQ в виде модуля ядра Linux, основываясь на имеющихся
описаниях в соответствующей литературе. 
Для выполнения цели работы необходимо выполнить следующие задачи:
\begin{enumerate}
    \item Проанализировать и сравнить дисциплины обслуживания PQ, CBQ, HTB, HFSC, FWFQ, CBWFQ.
    \item Восстановить алгоритмы Class-Based WFQ.
    \item Настроить среду для реализации и тестирования.
    \item Реализовать модуль ядра CBWFQ в ядре Linux.
    \item Реализовать интерфейс утилиты tc для управления модулем.
    \item Провести тестирование.
\end{enumerate}

Сравнительный анализ дисциплин обслуживания в работе основываются на научных статьях, документации и исходных кодах;
архитектура подсистемы Linux по управлению качеством обслуживания затрагивается в книге \cite{tcpip}, однако
в силу отсутствия полной документации все выводы о структуре делались на основе исходного кода ядра Linux.
Алгоритм CBWFQ воссоздавался на основе документации Cisco, книг \cite{tcpip} и \cite{Vagesna} и исходного кода
существующих дисциплин обслуживания. 


% Глава 1
\section{Постановка задачи, обзор существующей литературе по теме}
		\subsection{Постановка задачи}
		\subsubsection{Актульность задачи}

        Передача данных в компьютерных на сегодняшний день является основополагающим
        способом коммуникации не только людей, но и филиалов предприятий, государственных
        и военных структур. Важной задачей становится обеспечить должное качество и повысить
        вероятность передачи данных или в случае протоколов с гарантированной доставкой
        уменьшить количество попыток передать данные. Для этого выгодно обеспечивать на 
        каждом промежуточном пунке передачи данных наибольную вероятность того, что передача
        пройдёт успешно. Однако в силу физической ограниченности канала связи,
        временных затрат на обработку пакетов и отсутствия бесконечно больших буферов в
        промежуточных узлах возникают проблемы, приводящие к потере данных. Такие узлы
        на пути трафика называют "узким местом" (bottleneck). Они сильно влияют на 
        работу сети, из-за чего нужно уметь опреативно решать проблемы на подобных узлах
        для повышения эффективности всей сети.

		
		\subsubsection{Определение проблемы}

    	Устранить проблемы на медленных участках сети можно с помощью установки нового
    	и более мощного оборудования, однако это дорогостоящее решение и не всегда возможное.
    	Другим вариантом является возможность делить трафик на классы и устанавливать
    	требования к сети для каждого класса в отдельности. С помощью такой методики
    	появляется возможность понизить вероятность потери существенных пакетов, которые
    	могут являться причиной высокой нагрузки сети, и обеспечить должно обслуживание
    	каждого типа трафика в соответствии с требованиями.

    	Однако для поддержания такой возможности системы ядра IP-сети должны обладать
        возможностью дифференцирования и обслуживания различных типов сетевого трафика в
        зависимости от предъявляемых ими требований, однако Негарантированная доставко
    	данных не предполагает проведения какого-либо различия между потоками ифнормации
    	в сети, что препятствует передаче трафика, требующего выделения заданного минимума сетевых
        ресурсов и гарантии предоставления определенных услуг. Для разрешения описанной
        выше проблемы и было введено такое понятие, как качество обслуживания (quality of
        service -- QoS) в сетях IP. QoS предоставляет набор требований, предоставляемых к
    	ресурсам сети при транспортировке потока данных. [1]
    	Для учёта этих требований в узлах сети используются алгоритмы обслуживания пакетов,
    	называемые дисциплины обслуживания очередей (queuing disciplines, qdisc, ДО). Конфигурация
    	qdisc влияет на то, как выбирются пакеты из очередей пакетов (packet queues), какие
    	пакеты отбрасываются, как обрабатывается тот или иной тип трафика и так далее.[2]

    	Из вышесказанного следует, что дисциплины обслуживания очередей играют большую
    	роль в эффективной работе узла. Значит, чем более гибкая ДО используется, чем
    	больше нюансов трафика учитывает, тем точнее и эффективнее будут обрабатываться
    	пакеты, что повысит производительность узла и, следовательно, всей системы.

    	\subsubsection{Цель работы}

    	Цель данной работы заключается в реализации пакетного планировщика,
    	который будет предоставлять гибкие возможности для настойки управления трафиком и
    	его классификации. Платформой для реализации пакетного планировщика был выбрано Linux,
    	так как на его основе существует большое количество дистрибутивов, которые используются
    	в качестве сетевого ПО (Red Hat Enterprise Linux, Debian, Ubuntu Server, OpenWRT).




%	В качестве основы для реализуемого пакетного планировщика был выбран алгоритм CBWFQ.


%	\subsection{Описание алгоритма CBWFQ}
%
%    Class Based Weighted Fair Queueing (CBWFQ) расширяет стандартную
%	функциональность алгоритмa Weighted Fair Queueing (WFQ), предоставляя
%	возможность определять пользовательские классы трафика.
%
%	Для CBWFQ определяются классы, основываясь на критериях сопоставления, включая
%	протоколы, списки доступа и входной интерфейс. Пакеты, удовлетворяющие
%	критериям для класса составляют трафик этого класса. Очередь резервируется
%	для каждого класса, и трафик, принадлежащий классу, направляется в очередь
%	для этого класса. После определения класса указываются его характеристики,
%	такие как пропускная способность, вес, максимальное количество пакетов.
%	Весь неклассифицированный трафик попадает в единую очередь и получает
%	обслуживание по умолчанию.



% Глава 2
\section{Сравнительный анализ алгоритмов qdisc}
	% Сравнительный анализ алгоритмов qdisc
	\subsection{Дисциплины обслуживания очередей}

		%Понятие ДО. Зачем они и где они.обработки
		\textbf{Дисциплина обслуживания очередй (queuing discipline, qdisc)} -- это алгоритм выбора и обработки пакета
		из очереди пакетов, где под очередью подразумевается буфер пакетов. \\
		
%		ДО делятся на \textbf{приоритетные} и \textbf{бесприоритетные} и \textbf{классовые} и \textbf{бесклассовые}.
		ДО бывают следующих видов.
		\begin{description}
			\item[Приоритетные] -- ДО, которые назначают для разных видов трафика 
			\item[Бесприоритетные] -- ДО, которые не приоретизируют трафик. К таким ДО относится, к примеру, fifo.
			\item[Классовые]
			\item[Бесклассовые]
		\end{description}


	\subsection{Алгоритмы, реализованные в ядре Linux}

			\subsection{Бесклассовые алгоритмы (Classless)}
				\begin{itemize}
					\item \textbf{CODEL} (controlled-delay active queue management)
					\item \textbf{FIFO} (first in first out)
					\item \textbf{FQ} (fair queue)
					\item \textbf{RED} (random earky detection)
					\item \textbf{SFB} (stochastic fair blue)
				\end{itemize}

			\subsection{Классовые алгоритмы (Classful)}
				\begin{itemize}
               		\item CBQ (class based queueing) -- обработка пакетов в очередях 
               		\item HFSC (heirarchy fair service curve's control) --
               		\item HTB  (heirarchy token bucket) -- 
				\end{itemize}
			

%
%	\subsection{Методы управления трафиком в Linux}
%
%		Под управлением трафиком подразумевают набор механизов и операций, с помощью которого пакеты организовывают в очередь
%		для приёма/передачи через сетевой интерфейс; термин включает в себя механизмы принятия решений, какие пакеты принимать и
%		с какой скоростью на входящем интерфейсе, определения пакетов для передачи, их порядка и скорость передачи на исходящем устройстве.[1]
%
%		Рассказать, как решают проблемы с узкими местами
%		в маршруте трафика.



%
%	\subsection{Обзор существующих ДО в ядре Linux}
%
%		Какие есть, где используются, почему нужно ещё.
%
%        CLASSLESS
%        	* codel -- controlled-delay activa queue management
%        	* fifo	-- 
%        	* fq	-- fair queue
%        	* red   -- random earky detection 
%        	* sfb	-- stochastic fair blue 
%
%        CLASSFUL
%        	* cbq	-- class based queueing 
%        	* hfsc	-- heirarchy fair service curve's control under linux
%        	* htb	-- heirarchy token bucket
%
%	\subsection{Обоснование выбора}
%
%		Описать идею алгоритма CBWFQ
%
%		CBWFQ -- комбинация CBQ и WFQ
%
%		Описать приятности CBWFQ.
%
%	\subsection{Выводы}
%
%		Так как я не могу найти непосредственное описнание
%		алгоритма, придумаю его на основании всего того, что слышу.
%
%
%	\subsection{Список использованной литературы}
%		\begin{enumerate}
%			\item http://computerlib.narod.ru/html/traffic.htm, дата обращения 10-12-2017
%		\end{enumerate}


\section{Формальное описание выбранного алгоритма и его реализация}

\section{Заключение}

\section{Список использованной литературы}
	\begin{enumerate}
		\item Шринивас Вегешна, Качество обслуживания в сетях IP, 2003
		\item Иван Песин, Повесть о Linux и управлении трафиком, \\ http://computerlib.narod.ru/html/traffic.htm
	\end{enumerate}



\end{document}
