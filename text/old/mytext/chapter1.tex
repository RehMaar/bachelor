	\subsection{Постановка задачи}
		\subsubsection{Актульность задачи}

        Передача данных в компьютерных на сегодняшний день является основополагающим
        способом коммуникации не только людей, но и филиалов предприятий, государственных
        и военных структур. Важной задачей становится обеспечить должное качество и повысить
        вероятность передачи данных или в случае протоколов с гарантированной доставкой
        уменьшить количество попыток передать данные. Для этого выгодно обеспечивать на 
        каждом промежуточном пунке передачи данных наибольную вероятность того, что передача
        пройдёт успешно. Однако в силу физической ограниченности канала связи,
        временных затрат на обработку пакетов и отсутствия бесконечно больших буферов в
        промежуточных узлах возникают проблемы, приводящие к потере данных. Такие узлы
        на пути трафика называют "узким местом" (bottleneck). Они сильно влияют на 
        работу сети, из-за чего нужно уметь опреативно решать проблемы на подобных узлах
        для повышения эффективности всей сети.

		
		\subsubsection{Определение проблемы}

    	Устранить проблемы на медленных участках сети можно с помощью установки нового
    	и более мощного оборудования, однако это дорогостоящее решение и не всегда возможное.
    	Другим вариантом является возможность делить трафик на классы и устанавливать
    	требования к сети для каждого класса в отдельности. С помощью такой методики
    	появляется возможность понизить вероятность потери существенных пакетов, которые
    	могут являться причиной высокой нагрузки сети, и обеспечить должно обслуживание
    	каждого типа трафика в соответствии с требованиями.

    	Однако для поддержания такой возможности системы ядра IP-сети должны обладать
        возможностью дифференцирования и обслуживания различных типов сетевого трафика в
        зависимости от предъявляемых ими требований, однако Негарантированная доставко
    	данных не предполагает проведения какого-либо различия между потоками ифнормации
    	в сети, что препятствует передаче трафика, требующего выделения заданного минимума сетевых
        ресурсов и гарантии предоставления определенных услуг. Для разрешения описанной
        выше проблемы и было введено такое понятие, как качество обслуживания (quality of
        service -- QoS) в сетях IP. QoS предоставляет набор требований, предоставляемых к
    	ресурсам сети при транспортировке потока данных. [1]
    	Для учёта этих требований в узлах сети используются алгоритмы обслуживания пакетов,
    	называемые дисциплины обслуживания очередей (queuing disciplines, qdisc, ДО). Конфигурация
    	qdisc влияет на то, как выбирются пакеты из очередей пакетов (packet queues), какие
    	пакеты отбрасываются, как обрабатывается тот или иной тип трафика и так далее.[2]

    	Из вышесказанного следует, что дисциплины обслуживания очередей играют большую
    	роль в эффективной работе узла. Значит, чем более гибкая ДО используется, чем
    	больше нюансов трафика учитывает, тем точнее и эффективнее будут обрабатываться
    	пакеты, что повысит производительность узла и, следовательно, всей системы.

    	\subsubsection{Цель работы}

    	Цель данной работы заключается в реализации пакетного планировщика,
    	который будет предоставлять гибкие возможности для настойки управления трафиком и
    	его классификации. Платформой для реализации пакетного планировщика был выбрано Linux,
    	так как на его основе существует большое количество дистрибутивов, которые используются
    	в качестве сетевого ПО (Red Hat Enterprise Linux, Debian, Ubuntu Server, OpenWRT).




%	В качестве основы для реализуемого пакетного планировщика был выбран алгоритм CBWFQ.


%	\subsection{Описание алгоритма CBWFQ}
%
%    Class Based Weighted Fair Queueing (CBWFQ) расширяет стандартную
%	функциональность алгоритмa Weighted Fair Queueing (WFQ), предоставляя
%	возможность определять пользовательские классы трафика.
%
%	Для CBWFQ определяются классы, основываясь на критериях сопоставления, включая
%	протоколы, списки доступа и входной интерфейс. Пакеты, удовлетворяющие
%	критериям для класса составляют трафик этого класса. Очередь резервируется
%	для каждого класса, и трафик, принадлежащий классу, направляется в очередь
%	для этого класса. После определения класса указываются его характеристики,
%	такие как пропускная способность, вес, максимальное количество пакетов.
%	Весь неклассифицированный трафик попадает в единую очередь и получает
%	обслуживание по умолчанию.

