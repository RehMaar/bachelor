\documentclass[14pt, a4paper] {extarticle}
\usepackage[utf8] {inputenc}
\usepackage[T2A]{fontenc}
\usepackage[english, russian] {babel}
\usepackage[usenames,dvipsnames]{xcolor}

\usepackage{a4wide,longtable,amsmath,amsfonts,tikz}
% Таблицы
\usepackage{tabularx}
\usepackage{makecell}
\usepackage{multicol}

%Псевдокод
\usepackage{algpseudocode}
% Гиперссылки
\usepackage{hyperref}
% Рисунки
\usepackage{graphics}
% Для поиска и копипасти
\usepackage{cmap}
% Возможность переопределять оглавление и его стиль
\usepackage{tocloft}
\usepackage{etoolbox}
% Для вставки фигур
\usepackage{float}
\usepackage{subcaption}
\usepackage{caption}
\DeclareCaptionLabelSeparator{emdash}{ --- }
\captionsetup{labelsep=emdash}
\captionsetup{compatibility=false}
\setlength{\intextsep}{10mm}


% Правильные поля для диплома
\usepackage[top=20mm, bottom=20mm, left=25mm, right=10mm]{geometry}


\setlength\cftaftertoctitleskip{36pt}
\addto\captionsrussian{
  \def\figurename{{Рисунок}}
   \def\contentsname{{\hfill\bfseries\normalsize ОГЛАВЛЕНИЕ\hfill}}
}
\setcounter{tocdepth}{3} 

\usepackage{xcolor}
\usepackage{listings}
\lstset {
    basicstyle=\normalsize, 
    breaklines=true,
    tabsize=2,
    literate={--}{{-{}-}}2,
    captionpos=b
}
\lstdefinestyle{tcstyle}{emph={tc, qdisc, class, dev, parent, classid, filter,
							   protocol, flowid, cbwfq, default, bandwidth},
						 emphstyle = {\bfseries},
                         numbers=left,
                         stepnumber=1,    
                         firstnumber=1,
                         numberfirstline=true
}
\newcommand{\includecode}[3]{\lstinputlisting[caption=#3, escapechar=, style=custom#1]{#2}}

% Кастомный стиль подсветки для языка Си
\lstdefinestyle{customc}{
  belowcaptionskip=1\baselineskip,
  breaklines=true,
  frame=none,
  xleftmargin=\parindent,
  language=C,
  showstringspaces=false,
  basicstyle=\footnotesize,
  keywordstyle=\bfseries\color{green!40!black},
  commentstyle=\itshape\color{purple!40!black},
  identifierstyle=\color{black},
  stringstyle=\color{orange!40!black},
  %numbers=left,
  %stepnumber=1,    
  %firstnumber=1,
  %numberfirstline=true
}

% Красивый маркер ненумерованного списка в виде тире
\def\labelitemi{--}

% Enumeration
\usepackage{enumitem}
\setlist[enumerate]{topsep=0pt,itemsep=0ex,partopsep=1ex,parsep=1ex}
\setlist[itemize]{itemsep=0ex}

% Полуторный межстрочный интервал 
\usepackage[nodisplayskipstretch]{setspace}
\onehalfspacing

% Times New Roman
\usepackage{pscyr}
\renewcommand{\rmdefault}{ftm}

% Каждый пунтк оглавления с отточием
\usepackage{tocloft}
\renewcommand{\cftsecleader}{\cftdotfill{\cftdotsep}}

% Абзацный отступ равен 1.25 см
\parindent=1.25cm

% Номер страницы по середине верхнего поля
\usepackage{fancyhdr}
\pagestyle{fancy}
\fancyhf{}
\fancyhead[C]{\thepage}
\renewcommand{\headrulewidth}{0pt}

\fancypagestyle{plain}{\fancyhead[C]{\thepage}}


% Добавить абзацный отступ для первых абзацев в section/subsection,
% по умолчанию не добавляется
\usepackage{indentfirst}

% Обязательно переносить слова, чтобы соблюсти поля документа. Для
% соблюдения полей можно пренебречь правилами для тех слов и
% словосочетаний, о которых не знают словаря переносов (ruhyphen или
% ruenhyph). Оно почему-то работает. Взято с:
%
%   http://www.latex-community.org/forum/viewtopic.php?p=70342#p70342
%
\tolerance 1414
\hbadness 1414
\emergencystretch 1.5em
\hfuzz 0.3pt
\widowpenalty=10000
\vfuzz \hfuzz
\raggedbottom

% Формат заголовков
%   - Заголовок раздела по центру, кернингом побольше (отсебятина),
%     прописными буквами, выделено жирным, X   ЗАГОЛОВОК
%   - Заголовок подраздела и "пункта" со смещением, как у абзаца, по
%     левому краю, выделено жирным, X.Y[.Z]   Заголовок
\usepackage{titletoc}
\usepackage{titlesec}
\titleformat{\section}[block]{\centering\bfseries\normalsize}
                         {\arabic{section}}{1ex}{\MakeUppercase}
\titleformat{\subsection}[block]{\hspace{\parindent}\bfseries\normalsize}
                         {\arabic{section}.\arabic{subsection}}{1ex}{}
\titleformat{\subsubsection}[block]{\hspace{\parindent}\bfseries\normalsize}
                         {\arabic{section}.\arabic{subsection}.\arabic{subsubsection}}{1ex}{}


% TODO: 14pt * 3 = 42pt (три интервала до и после)
\titlespacing*{\section}{0pt}{42pt}{42pt}

\usepackage[square,numbers]{natbib}
\bibliographystyle{unsrtnat}

\urlstyle{same}

\begin{document}
    \setcounter{figure}{0}
    \newcommand{\mc}[0]{\makecell}
    \newcommand\setrow[1]{\gdef\rowmac{#1}#1\ignorespaces}
    \newcommand\clearrow{\global\let\rowmac\relax}
    \clearrow
    \setcounter{page}{3}

   \phantomsection
   \tableofcontents

	% Введение
	\newpage
	\phantomsection
\section*{ВВЕДЕНИЕ}
\addcontentsline{toc}{section}{ВВЕДЕНИЕ}

В наши дни на фоне высокого спроса на высокоскоростную и надёжную передачу данных
остро встаёт проблема обработки трафика и обеспечения должного уровня обслуживания
в узлах компьютерных сетей. Крупные производители сетевого оборудования
предоставляют эффективные решения обозначенных проблем, однако оборудование и программное
обеспечение стоят немалых денег, что делает их недоступными для массового пользователя.
Большое распространение в качестве сетевого ПО получили дистрибутивы Linux,
которые из-за многолетнего использования имеют богатый набор возможностей в этой области.
Таким образом, встаёт вопрос о интеграции возможностей,
предоставляемых гигантами индустрии, с широко распространённым открытым вариантом.

В связи с этим было решено изучить дисциплину обслуживания, разработанную компанией Cisco,
известным вендором на рынке сетевых решений, и внедрить её в ядро последней стабильной версии Linux. 
В качестве дисциплины обслуживания был выбран взвешенный алгоритм честного
обслуживания основанного на классах (Class-Based Weighted Fair Queuing, CBWFQ).

CBWFQ является расширением функциональности широко известного взвешенного алгоритма
честного обслуживания очереди (Weighted Fair Queuing). Он поддерживает
определение пользовательских классов трафика на основе ряда критериев соответствия
(протокол, входящий интерфейс и т.д.) и назначение их характеристик, которые
отвечают за выделяемые классу ресурсы (вес, пропускная способность, максимальное
количество пакетов в очереди, задержка). Такой подход предоставляет гибкую настройку
распределения пропускной способности канала между классами трафика и оказывается весьма
эффективным в передаче данных в сравнении с рядом других популярных дисциплин.\cite{thesis}

\textbf{Цель работы} -- реализовать алгоритм Class-Based WFQ в виде модуля ядра Linux, основываясь на имеющихся
описаниях в соответствующей литературе. 
Для выполнения цели работы необходимо выполнить следующие задачи:
\begin{enumerate}
    \item Проанализировать и сравнить дисциплины обслуживания PQ, CBQ, HTB, HFSC, FWFQ, CBWFQ.
    \item Восстановить алгоритмы Class-Based WFQ.
    \item Настроить среду для реализации и тестирования.
    \item Реализовать модуль ядра CBWFQ в ядре Linux.
    \item Реализовать интерфейс утилиты tc для управления модулем.
    \item Провести тестирование.
\end{enumerate}

Сравнительный анализ дисциплин обслуживания в работе основываются на научных статьях, документации и исходных кодах;
архитектура подсистемы Linux по управлению качеством обслуживания затрагивается в книге \cite{tcpip}, однако
в силу отсутствия полной документации все выводы о структуре делались на основе исходного кода ядра Linux.
Алгоритм CBWFQ воссоздавался на основе документации Cisco, книг \cite{tcpip} и \cite{Vagesna} и исходного кода
существующих дисциплин обслуживания. 



	% Chapter 1: сревнение с другими штуками
	\newpage
	\section{АНАЛИЗ ДИСЦПЛИН ОБСЛУЖИВАНИЯ ЯДРА LINUX}

    \subsection{Дисциплины обслуживания}

%    Добавить описание ДО. Рассказать про классовые и безклассовые.
%    Описать главные характеристики, которые будут в таблице.

	Дисциплина обслуживания очередей (ДО) -- правило выбора заявок
	из очереди для обслуживания\cite{Aliev}.

	Возможно, нужно описать некоторые термины, которые потом будут представлены в таблице.

    \subsection{Приоритетные очереди}

    Приоритетные очереди (Priority Queueing, PQ) -- это техника обслуживания,
    при которой используется множество очередей с разными приоритетами. Очереди
    обслуживаются в циклическом порядке (алгоритмом round-robin) от самого высокого 
    до самого низкого приоритета; обслуживание следующей по порядку очереди происходит,
    если более приоритетные очереди пусты. Каждая очередь внутри обслуживается в порядке FIFO
    (First-In, First-Out). В случае переполнения отбрасываются пакеты из очереди
    с более низким приоритетом.\cite{packethandling}
    
    Дисциплина используется, чтобы понизить время отклика, когда нет нужды замедлять трафик\cite{tcprio}. 

    В Linux алгоритм реализован в виде дисцплины prio, которая создаёт фиксированное
    значение очередей обслуживания, управляемые дисциплиной pfast\_fifo, и управляет
    очередями в соответствии с картой приоритетов.\cite{tcprio}

    \begin{figure}[ht!]
        \center
        \includegraphics{pdfimages/pq.pdf}
        \caption{Схема обслуживания алгоритмом приоритетных очередей}
    \end{figure}

    Приемущества алгоритма состоят в следующем:
    \begin{itemize}
		\item возможность понижения времени отклика, когда нет необходимости замедлять трафик \cite{tcprio};
        \item наиболее простая в реализации классовая дисциплина обслуживания;
        \item для software-based маршрутизаторов PQ предоставляет относительно небольшую
             вычислительную нагрузку на систему в сравнении с более сложными ДО;
        \item PQ позволяет маршрутизаторам организовывать буферизацию пакетов и обслуживать
             один класс трафика отдельно от других. \cite{suppdiff}
    \end{itemize}

    Однако приоритетные очереди обладают рядом существенных недостатков.
    \begin{itemize}
        \item возникает проблема простоя канала (отсутствие обслуживания в течение продолжительного времени)
			  для низкоприоритетного трафика при избытке высокоприоритетного\cite{packethandling};
        \item избыточный высокоприоритетный трафик может значительно увеличивать
                задержку и джиттер для менее приоритетного трафика;
        \item не решается проблема с TCP и UDP, когда TCP-трафику даётся высокий приоритет и он
                пытается поглотить всю пропускную способность. \cite{suppdiff}
    \end{itemize}

    \subsection{Алгоритм управления очередями на основе классов}

        Алгоритм управления очередями на основе классов (Class Based Queueing, CBQ) -- это
        классовая дисциплина обслуживания, которая реализует
        иерархическое разделение канала между классами, и позволяет
        шейпинг трафика. \cite{tccbq}

        Главная цель CBQ -- это планировка пакетов в очередях, гарантия определённой
        скорости передачи и разделение канала. Если в очереди нет пакетов, её
        пропускная способность становится доступной для других очередей. Сила
        этого метода состоит в том, что он позволяет справляться со значительно
        различными требованиями к пропускной способности канала среди потоков. Это
        реализовано путём назначения определённого процента доступной ширины
        канала каждой очереди. CBQ избегает проблему простоя канала, которой страдает
        алгоритм PQ, так как как минимум один пакет обслуживается от каждой очереди
        в течение цикла обслуживания.\cite{packethandling}

		Алгоритм CBQ представляет канал в кажется иерархической структуры \cite{linksharing},
		пример которой представлен на Рисунке~\ref{pic:cbq}. Голубым цветом обозначен узел,
		представляющий собой основной канал; он разделяется между двумя классами трафика:
		интеркативным (левый узел) и остальным (правый узел), -- которым назначается процент
		пропускной способности от остального канала. Весь трафик, причиляемый к классу,
		будет получать выделенную пропускную способность для этого класса. Эти
		классы трафика могут разделяться на подклассы и так далее. Если класс не использует
		пропускную способность, она будет выделяться классу-соседу. Этот механизм
		называется механизмом разделения канала \cite{linksharing}. 

        \begin{figure}[ht!]
			\center
            \includegraphics[scale=1.2]{./pdfimages/cbq.pdf}
            \caption{Разделение канала при CBQ}
			\label{pic:cbq}
        \end{figure}

        Алгоритм CBQ состоит в следующем. Сначала пакеты классифицируются в классы
        обслуживания в соответствии с определёнными критериями и сохраняются в
        соответствующей очереди. Очереди обслуживаются циклически. Различное
        количество пропускной способности может быть назначено для каждой очереди
        двумя различными способами: с помощью позволения очереди отправлять более
        чем один пакет на каждый цикл обслуживания или с помощью позволения очереди
        отправлять только один пакет за цикл, но при этом очередь может быть обслужена
        несколько раз за цикл.\cite{packethandling}

		Вычисления в CBQ основываются на вычислении времени в микросекундах между
		запросами, на основе которого рассчитывается средняя загруженность канала;
		в этом и состоит главная проблема неточности CBQ в Linux.\cite{lartc} 

        Преимущества алгоритма состоят в следующем:
        \begin{itemize}
            \item позволяет контролировать количество пропускной способности для каждого
                  класса обслуживания;
            \item каждый класс получает обслуживание, вне зависимости от других классов. Это
                  помогает избегать проблемы PQ, когда при избытке высокоприоритезированного
                  трафика низкоприоритезированный не обслуживался вообще.\cite{packethandling}
        \end{itemize}

        Недостатки же в большей следуют из особенностей реализации алгоритма в системе Linux:
        \begin{itemize}
            \item честное выделение пропускной способности происходит, только если
                  пакеты из всех очередей имеют сравнительно одинаковый размер. Если один класс
                  обслуживания содержит пакет, который длиннее остальных, этот класс обслуживания
                  получит большую пропускную способность, чем сконфигурированное значение \cite{packethandling};

            \item высокая сложность реализации. В ядре Linux реализация СBQ приближённая и
                  в некоторых случаях может давать неверные результаты.\cite{lartc}
        \end{itemize}

    \subsection{Алгоритм иерархического маркерного ведра}

        Алгоритм иерархического маркерного ведра (Hierarchical Token Bucket, HTB) -- дисциплина
        обслуживания с иерархическим разделением канала между классами.

		HTB, подобно CBQ, использует механизм разделения канала. 
		HTB обеспечивает, что количество обслуживания, предоставляемое каждому классу, является,
		минимальным значением из запрошенного количества и назначенного классу. Когда класс
		запрашивает меньше, чем ему выделенно, оставшаяся пропускная способность распределяется между
		другими классами, которые требуют обслуживание.\cite{htb}

		Отличительная особенность HTB от CBQ состоит в том, что в HTB принцип работы
		основывается на определении объема трафика\cite{lartc}, что даёт более точные результаты.

		HTB состоит из произвольного числа иерархически организованных фильтров маркерного
        ведра (Token Bucket Filter, TBF)\cite{packethandling}, однако реализация не использует готовый
		модуль \texttt{tbf}: алгоритм маркерного ядра втроен в код реализации HTB,	
		что повышает его эффективность. Внутренние классы содержат фильтры, которые
		распределяют пакеты по очередям и метаинформацию, позволяющую функционировать
		механизм разделения канала. Листевые классы содержат очереди, которые содержат
		очереди, которые управляются сконфигурированными дисциплинами обслуживания (по
		умполчанию pfifo\_fast).[Исходный код] Пример сконфигурированного дерева HTB
		представлен на Рисунке~\ref{pic:htb_hier}.
		
        \begin{figure}[ht!]
            \center
            \includegraphics{./pdfimages/class_hierh_htb.pdf}
            \caption{Пример иерархии классов при использовании дисицплины HTB}
			\label{pic:htb_hier}
        \end{figure}

        При добавлении пакета в очередь HTB начинает обход дерева от корня
        для определения подходящей очереди: в каждом узле происходит поиск
        инструкций, и затем происходит переход в узел, на который ссылается
        инструкция. Обход заканчивается, когда алгоритм доходит до листа,
        в очередь которого помещается пакет.\cite{tchtb} В реализации алгоритма
		существует прямая очередь, которая используется не только в качестве
		очереди с наивысшем приоритетом, но и как очередь, в которую попадают
		пакеты, не определённые в другую очередь. Это мера не самая удачная,
		но используется для избежания ошибок.[Исходный код]

        Приемущества алгоритма HTB приведены ниже.
        \begin{itemize}
            \item Наиболее используемая дисциплина обслуживания в Linux, так как HTB эффективно
				  справляется с обработкой пакетов, а конфигурация HTB легко масштабируется.\cite{lartc}
            \item Иерархическая структура предоставляет гибкую возможность конфигурировать трафик.
            \item Не зависит от характеристик интерфейса и не нуждается в знании о лежащей в
                  основе пропускной способности выходного интерфейса из-за свойств TBF. \cite{tchtb}
            \item Вычислительно проще, чем алгоритм CBQ.\cite{htb}
        \end{itemize}

        Недостатки.
        \begin{itemize}
            \item Медленее CBQ в N раз, где N -- глубина дерева разделения, что, однако, компенсируется простотой вычислений.\cite{htb}
			\item Нужно что-то ещё весомое.
        \end{itemize}

    \subsection{Алгоритм иерархических честных кривых обслуживания}
% https://www.cs.cmu.edu/~hzhang/papers/SIGCOM97.pdf
% http://linux-tc-notes.sourceforge.net/tc/doc/sch_hfsc.txt
% https://serverfault.com/questions/105014/does-anyone-really-understand-how-hfsc-scheduling-in-linux-bsd-works

        HFSC -- Hierarchical Fair-Service Curve -- иерархический алгоритм планирования пакетов,
        основанный на математической модели честных кривых обслуживания (Fair Service Curve),
        где под термином "кривая обслуживания" подразумевается зависящая от времени
        неубывающая функция, которая служит нижней границей количества обслуживания,
        предоставляемого системой.\cite{hfsc}

        HFSC ставит перед собой цели:
        \begin{itemize}
            \item гарантировать точное выделение пропускной способности и задержки для всех листовых классов (критерий реального времени);
            \item честно выделять избыточную пропускную способность как указано классовой иерархией (критерий разделения канала);
            \item минимизировать несоответствие кривой обслуживания идеальной модели и действительного количество обслуживания.\cite{tchfsc}
        \end{itemize}

        Алгоритм планировки основан на двух критериях: критерий реального времени
        (real-time) и критерий разделения канала (link-sharing). Критерии реального времени
        используются для выбора пакета в условиях, когда есть потенциальная опасность,
        что гарантия обслуживания для листового класса нарушается. В ином случае
        используется критерий разделения канала.\cite{tchfsc}

		\begin{figure}[ht!]
		 	\center
            \begin{subfigure}[b]{0.3\textwidth}
                \includegraphics[width=\textwidth]{pdfimages/hfsc_concave.pdf}
                \caption{Вогнутая кривая.}
                \label{pic:hfsc_concave}
            \end{subfigure}
            \hspace{0.2\textwidth}
            \begin{subfigure}[b]{0.3\textwidth}
                \includegraphics[width=\textwidth]{pdfimages/hfsc_convex.pdf}
                \caption{Выпуклая кривая.}
                \label{pic:hfsc_convex}
            \end{subfigure}
			\caption{Примеры кривых обслуживания. $m_1$ -- скорость в стационарном состоянии,
			$m_2$ -- скорость в режиме burst, $d$ -- время, за которое происходит передача
			в режиме burst.\cite{hfscguide}}
			\label{pic:hfscline}
        \end{figure}

		На Рисунке~\ref{pic:hfscline} изображены примеры кривых обслуживания, используемых
		в дисциплине HFSC; параметры $m_1$, $m_2$ и $d$, отображённые на графиках, задаются
		при конфигурации дисциплины.\cite{hfscguide}

        HFSC использует три типа временных параметров: время крайнего срока (deadline
        time), "подходящее" время (eligible time) и виртуальное время (virtual time). Время крайнего
        срока назначается таким образом, чтобы, если крайние сроки всех пакетов сессии
        выполнены, его кривая была гарантирована. "Подходящее" время используется для
        выбора критерия планировки для следующего пакета. Виртуальное время показывает
        нормализованное количество обслуживания, которое получил класс. Виртуальное
        время присуще всем вершинам дерева классов, так как является важным параметром
        при критерии разделение канала, при котором должно минимизироваться
        несоответствие между виртуальным временем класса и временами его соседей
        (так как в идеальной модели виртуальное время соседей одинаково); при выборе
        критерия разделения канала алгоритм рекурсивно, начиная с корня, обходит всё
        дерево, переходя в вершины с наименьшим виртуальным временем. Время крайнего
        срока и <<подходящее>> время используются дополнительно в листьевых классах,
        так как в этих вершинах непосредственно содержатся очереди.\cite{hfsc}

		Основное преимущество алгоритма состоит в том, что он основан
		на формальной модели с доказанными нижними границами. Он даёт гаранитрованные
		результаты и вычисляет более точно, чем дисциплины CBQ и HTB, которые
		служат схожим целям.

		Главные же недостатки HFSC заключены в его достоинстве. Алгоритм основан
		на формальной модели и имеет множество параметров, требующих дополнительных
		расчётов и времени на подготовку к конфигурации. Также он довольно сложен
		в реализации и поддержке. 

    \subsection{Взвешенный алгоритм честного обслуживания очередй на основе потоков}

    WFQ (Weighted Fait Queueing) -- динамический метод планировки пакетов, который
    предоставляет честное разделение пропускной способности всем потокам трафика.
    WFQ применяет вес, чтобы идентифицировать и классифицировать трафик
    в поток и определить, как много выделить пропускной способности каждому
    потоку относительно других потоков. WFQ на месте планирует интерактивный трафик в начало очереди,
    уменьшая там самым время ответа, и честно делит оставшуюся пропускную
    способность между остальными потоками. \cite{ciscoguide}
    % https://www.cisco.com/c/en/us/td/docs/ios/12_2/qos/configuration/guide/fqos_c/qcfconmg.html

   
    \begin{figure}[ht!]
			\center
        \includegraphics{./pdfimages/fwfq.pdf}
        \caption{Схема WFQ системы на основе потоков}
    \end{figure}

    %В соответствии с механизмом WFQ вес пакета определяется на основании значения поля
    %IP-приоритета в заголовке пакета $\text{вес} = 4096 \div (\text{IP-приоритет} + 1)$
    %(в новых версиях $32768 \div (\text{IP-приоритет} + 1)$). 
    %Вес пакета строго зависит от его приоритета и не может
    %быть изменён.

    Планировщик не нарушает порядка обработки пакетов,
    принадлежащих одному потока, даже в том случае, если они имеют различный приоритет.
    C этой целью поток реализуется в виде хэша, определяемого IP-адресом источника,
    IP-адресом цели назначения, полем протокола IP, номерами портов TCP/UDP и пятью
    битами байта ToS (Type of Service).Очередь потока обслуживается в соответствии с алгоритмом FIFO.

    % Алгоритм WFQ отбрасывает пакеты только наиболее активных потоков трафика.

    WFQ на основе потока использует для обработки каждого трафика так называемые
    очереди диалога (conversation queue).
    Поскольку память -- конечный ресурс, число очередей диалога по умолчанию
    ограничено 256. Если число потоков превысит число очередей, допускается
    использование одной очереди для обработки нескольких потоков.\cite{Vagesna}

    В целях планировки в WFQ длина очереди измеряется не в пакетах, а во времени,
    которое заняла бы передача всех пакетов в очереди. WFQ адаптирует количество
    потоков и выделяет одинаковое количество полосы пропускания каждому потоку.
    Поток с маленькими пакетами, которые обычно являются интерактивными потоками,
    получают лучшее обслуживание, потому что они не нуждаются в большой полосе пропускания;
    также они получаются низкую задрежку, потому что у меньших пакетов меньшее
    время отправки (finish time). Время отправки -- это сумма текущего времени и
    время, которое заняла бы отправка пакета. Текущее время ноль, если в очереди
    нет пакетов. WFQ поместит пакет в аппаратную очередь, основываясь на времени отправки в
    порядке возрастания.[И ОТКУДА ЭТО?]

    Чтобы ввести вес в расчёт то, в каком порядке будут обслуживаться очереди,
    WFQ использует время окончания и приоритет IP (IP precedence). Вес расчитывается
    как время окончания, делённое на приоритет IP плюс один (во избежание деления
    на ноль). Однако для увеличения производительности в маршрутизаторах Cisco
    взамен времени отправки (finish time) используется размер пакета, так как он
    пропорционален времени; к тому же деление на приоритет IP заменяется на умножение
    фиксированного значения, просчитанного заранее (это сделано из-за того,
    что деление более трудная операция для CPU, чем умножение).\cite{Vagesna}

    WFQ использует два метода отбрасывания пакетов: ранее (Early Dropping) и агрессивное
    (Aggressive Dropping) отбрасывания. Ранее отбрасывание срабатывает тогда, когда
    достигается congestive discard threshold (CDT); CDT -- это количество пакетов, которые могут
    находиться в системе WFQ перед тем, как начнётся отбрасывание новых пакетов
    из самой длиной очереди; используется, чтобы начать отбрасывание пакетов
    из наиболее агрессивного потока, даже перед тем, как достигнется предел
    hold queue out (HQO). HQO -- это максимальное количество пакетов, которое может быть
    во всех выходящих очередях в интерфейсе в любое время; при достижении HQO
    срабатывает агрессивный режим отбрасывания. Алгоритм представлен на Рисунке~\ref{pic:wfqdropalgo}.
    \cite{wfqdrop}

    \begin{figure}[ht!]
			\center
        \includegraphics{./pdfimages/fwfq_drop.pdf}
        \caption{Схема отбрасывания пакетов WFQ.}
		\label{pic:wfqdropalgo}
    \end{figure}

    Приемущества WFQ.
    \begin{itemize}
        \item Простая конфигурация.
        \item Отбрасывание пакетов из более агрессивных потоков.
		\item Честное обслуживание.
    \end{itemize}

    WFQ страдает от нескольких недостатков.
    \begin{itemize}
        \item Трафик не может регулироваться на основе пользовательски определённых классов обслуживания.
        \item Не поддерживает задание определённой пропускной способности для типа трафика.
		\item В Cisco системах WFQ поддерживается только на медленных каналах.\cite{wfqdis}
    \end{itemize}

    Эти ограничения были исправлены CBWFQ.

    \subsection{Взвешенный алгоритм честного обслуживания очередей на основе классов}

    % https://www.cisco.com/en/US/docs/ios/12_0t/12_0t5/feature/guide/cbwfq.html
%   Надо описать, что было создано в Cisco и т.п. и как это работает в циско.
%   Таблица сравнений с flow-based WFQ
%   Предоставляет одну очередь для класса, всего 64 класса.
%   Позволяет указывать пропускную способность для класса
%   Предоставляет гарантию пропускной способности для пользовательких классов.
%   Предоставляет поддержку flow-based WFQ для не определёнными пользователями
%   трафиков класса.
%   Требует конфигурацию.
%   [cм ссылку в комментах]
    % https://www.cisco.com/c/en/us/td/docs/ios/12_2/qos/configuration/guide/fqos_c/qcfconmg.html

    CBWFQ (Class-based weighted fair queueing) -- основанный на классах взвешенный алгоритм равномерного обслуживания 
    очередей[вагешен?]; является расширением функциональности дисциплины обслуживания WFQ,
    основанной на потоках, для предоставления определяемых пользователями классов трафика. 

    Сlass-Based WFQ -- это мехнизм, использующийся для гарантировании пропускной способности
    для класса. Для CBWFQ класс трафика определяется на основе заданных криетриев
    соответствия: список контроля доступа (ACL), протокол, входящий интерфейс и т.п. Пакеты,
    удовлетворяющие криетриям класса, составляют трафика для этого класса. Дисциплина
    позволяет задавать до 64-х пользовательских классов.

    После определения класса, ему назначаются характеристики, которые определяют
    политику очереди: пропускная способность, выделенная классу, максимальная
    длина очереди и так далее. Алгоритм CBWFQ позволяет явано указать требуемую минимальную
    полосу пропускания для каждого класса трафика. Полоса пропускания используется
    в качестве веса класса. Вес можно задать в абсолютной (опция \texttt{bandwidth}),
    в процентной (опция \texttt{bandwidth percent}) и в доле от оставшиейся
    полосы пропускания (опция \texttt{bandwidth remaining precent}) величинах.
    Кроме пользовательских классов CBWFQ предоставляет стандартный класс (default class),
    в который попадает весь трафик, который не был классифицирован. В стандартном классе
    управление очередью может осуществляться с помощью алгоритмов FIFO и FQ (Fair Queueing). \cite{ciscoguide} 

    В случае переполнения очередей начинает работать алгоритм отбрасывания пакетов.
    В качестве политики отбрасывания пакетов по умолчанию используется отбрасывание конца
    очереди (Tail Drop), однако допускается сконфигуривать работу
    алгоритм взвешенного произвольного раннего обнаружения (Weighted Random Early Detection, WRED)
    для каждого класса.\cite{ciscoguide}

 	\begin{figure}
		\center
    	\includegraphics[scale=1.1]{pdfimages/cbwfq.pdf}
		\caption{Схема CBWFQ}
	\end{figure}   

	Преимущества:
	\begin{itemize}
		\item позволяет явно задать полосу пропускания для класса;
		\item позволяет содавать классы трафика и настраивать их в соответствии с требованиями;
		\item простая конфигурация вследствие небольшого числа параметров.\cite{Vagesna}\cite{ciscoguide}
	\end{itemize}

	Недостатки:
	\begin{itemize}
		\item нет поддержки работы с интерактивным трафиком (что исправляется в дисциплине обслуживания Low Latency Queueing (LLQ),
		которая является развитием CBWFQ);
		\item в Cisco реализации наблюдается ограничение на количество пользовательских классов (до 64-х классов);\cite{Vagesna}
		\item отсутствие открытой реализации, что усложняет реализацию алгоритма в других системах и требует его полного воссоздания
		на основе имеющихся источников.
	\end{itemize}

	\subsection{Вывод}

	Каждая из рассмотренных ДО обладает своими достоинствами и недостатками. В Таблице~\ref{tab:compqdisc}
	приведено сравнение основных элементов дисциплин обслуживания.

	...

	Поэтому реализация CBWFQ в ядре Linux целесообразна.

	\begin{table}[ht!]
		\center

        \begin{tabular}{|>{\rowmac}c|>{\rowmac}c|>{\rowmac}c|>{\rowmac}c|>{\rowmac}c|>{\rowmac}c|>{\rowmac}c<{\clearrow}|}
            \hline
            \setrow{\bfseries}     Свойство     & PQ   & CBQ   & HTB   & HFSC  & FWFQ  & CBWFQ \\ \hline
            {\bf \mc{Метод\\ планирования     }}& RR   & WRR   & RR    & RT/LS & WFQ   & WFQ   \\ \hline
            {\bf Честность                     }& -    & -     & -     & +     &  +    &  +    \\ \hline
            {\bf Отбрасывание                  }& TD   & TD    & TD    & TD    & ED/AD & TD/WRED \\ \hline
            {\bf \mc{Разделение\\ канала      }}& -    &  +    &  +    &  +    &  -    &  -    \\ \hline
            {\bf \mc{Сложность \\ реализации  }}& Н    & В     &С      & В     & С     & С \\ \hline
            {\bf \mc{Сложность \\ конфигурации}}& Н    & В     &С      & В     & Н     & Н \\ \hline
            {\bf \mc{Конфигурация\\ классов   }}& -    & +     & +     & +     & -     & + \\ \hline
            {\bf \mc{Реализация\\ в Linux     }}& +    & +     & +     & +     & -     & -  \\ \hline
        \end{tabular}
    	\caption{Сравнительная таблица дисциплин обслуживания. Обозначения: RR -- Round Robin (алгоритм циклического
    	обслуживания), WRR -- Weighted Round Robin (алгоритм взвешенного циклического обслуживания), RT/LS --
    	Real-Time/Link-Sharing (алгоритм, который обслуживает очередь в зависимости от критерия реального времени
    	и критерия разделения канала), TD -- Tail Drop (алгоритм обрасывания "хвостов"), ED/AD --
    	Early-Detection/Aggressive-Detection (алгоритм раннего и агрессивного обнаружения),
    	WRED -- Weighted Random Early Detection (алгоритм взвешенного раннего обнаружения),
    	H -- низная сложность, C -- средняя сложность, В -- высокая сложность (сложность
    	оценивалась в относительно рассмотренных дисциплин).}
		\label{tab:compqdisc}
	\end{table}



	% Chapter 2: Описание реализации и тестирования
	\newpage
	\section{РЕАЛИЗАЦИЯ CLASS-BASED WFQ В ЯДРЕ LINUX}

	\subsection{Описание устройства подсистемы планировки в ядре Linux}
	% Информация из комментариев к ним.
	% Файлы: linux/net/sched/sch_api.c, linux/include/net/{sch_generic.h,pkt_sched.h}. 

	В операционной системе Linux
	дисциплина обслуживания, обозначемая термином qdisc, используется
	для выбора пакетов из выходящей очереди для отправки на выходной интерфейс.
	Схема движения пакета приведена на Рисунке~\ref{pic:flow}. Выходная очередь
	обозначена термином egress; именно на этом этапе следования пакета
	и работает механизм qdisc.\cite{lartc}

    \begin{figure}[ht!]
        \center
        \includegraphics[scale=1.3]{pdfimages/qdisc.pdf}
        \caption{Схема движения пакета в системе Linux\cite{tcpip}}
		\label{pic:flow}
    \end{figure}

	В общем случае, дисциплина обслуживания -- это чёрный ящик, который может
	принимать поток пакетов и выпускать пакеты,
	когда устройство
	готово к отправке, в порядка и во время, определёнными спрятанным в ящике
	алгоритмом. В ядре Linux дисциплины обслуживания представляются в качестве
	модулей ядра, которые реализуют предоставляемый ядром интерфейс.

	Linux поддерживает классовые и бесклассовые дисцплины обслуживания. Примером
	бесклассовой дисциплины служит pfifo\_fast, классовой -- htb.\cite{lartc}

	Классы представляют собой одельные сущности в иерархии основной дисциплины.
	Если структура представляет собой дерево, то в классах-узлах могут содержаться
	фильтры, которые определят пакет в нужный класс-потомок. В классах-листьях
	непосредстенно располагаются очереди, которые управляются внутренней дисциплной
	обслуживания. По умолчанию это pfifo\_fast, но можно назначить другие. 

	Каждый интерфейс имеет корневую дисцплину, которой
	назначается идентификатор (handle), который используется для обращения к дисциплине.
	Этот идентификатор состоит из двух частей: мажорной (MAJ) и минорной (MIN); мажорная
	часть определяет родителя, минорная -- непосредственно класс. На Рисунке~\ref{pic:clheirh}
	представлен пример иерархии.

	\begin{figure}[ht!]
		\centering
		\includegraphics[scale=1.3]{./pdfimages/class_hierh.pdf}
		\caption{Схема классовой иерархии с использованием идентификаторов MAJ:MIN}
		\label{pic:clheirh}
	\end{figure}

	Идентификатор класса называется classid (к примеру, 1:10),
	а идентификтор его родителя -- parenid (1:1 для классов 1:10 и 1:20). По этим
	идентификаторам происходит поиск нужного класса внутри дисциплины.

	Такая иерархия позволяет организовать гибкую систему классификации с набором классов
	и их подклассов, пакеты в которые назначются фильтрами, которые предоставляются ядром.

	%CBWFQ является классовой дисциплиной обслуживания с конфигурируемыми классами (в отличие от
	%дисциплины PQ) и классом по умолчанию. Ядро предоставляет ряд полезных функций и структур
	%данных, которые значительно упрощают написания необходимого кода в ядре и избавляет от
	%совершения ряда критических ошибок. Таким образом, для наиболее эффективной реализации
	%дисципилны обслуживания CBWFQ следует использовать предоставляемые ядром функции.

	\subsection{Интерфейс управления трафиком}

	В Linux управление трафиком осуществляется с помощью подсистемы Traffic Control,
	которая предоставляет пользовательский интерфейс с помощью утилиты \texttt{tc}.
	\texttt{tc} -- это пользовательская программа, которая позволяет настраивать
	дисцплины обслуживания в Linux. Она использует Netlink в качестве
	коммуникационного канала для взаимодействия между пользовательским
	пространством и пространством ядра. \texttt{tc} добавляет новые дисциплины
	обслуживания, классы трафика, фильтры и предоставляет команды для
	управление всеми обозначенными объектами.\cite{tcpip}

	\texttt{tc} предоставляет интерфейс для дисцплины обслуживания,
	представленный структурой \lstinline{struct qdisc_util}, которая
	описывает функции для отправления команд и соответствующих параметров ядру
	и вывода сообщений о настройки дисциплины, списках классов и их настройки, а
	также статистику от ядра. Сообщение, помимо общей информации для подсистемы,
	содержит специфичную для дисциплины структуру с опциями, описываемую
	в заголовке ядра (pkt\_sched.h). 

	Для назначения новой дисциплины обслуживания на интерфейс используется
	команда \lstinline{tc qdisc add} системными параметрами (к примеру, название
	интерфейса), названием дисциплины и её локальными параметрами, которые
	определяются и обрабатываются в модуле дисциплины для утилиты \texttt{tc}. 
	Для внесения измнений и удаления используются соответственно \lstinline{tc change}
	и \lstinline{tc delete}.

	Для классовых дисциплин используется команда \lstinline{tc class} с подкомандами
	\lstinline{add}, \lstinline{change} и так далее. Классы обычно имеют параметры,
	отличные от параметров всей дисциплины обслуживания, поэтому нуждаются в отдельной
	структуре данных и функции обработчике.

	Таком образом, для использования дисциплины обслуживания необходимо
	реализовать интерфейс в системе \texttt{tc}. Патчи для утилиты \texttt{tc}
	и ядра Linux, обеспечивающие взаимодействие между пользовательским пространством
	и дисциплиной представлен в Приложении А. 

	\subsection{Описание интерфейса}

	API ядра для подсистемы qdisc предосавтляет две функции: \lstinline{register_qdisc(struct Qdisc_ops *ops)}
	и обратную -- \lstinline{unregister_qdisc(struct Qdisc_ops *ops)}, которые регистрируют
	и разрегистрируют дисциплину обслуживания на интерфейсе. Важно отметить, что обе эти
	функции принимают в качестве аргумента структуру \lstinline{struct Qdisc_ops},
	которая явным образом идентифицирует дисциплину обслуживания в ядре.\cite{linuxsrc}

	Структура \lstinline{struct Qdisc_ops} помимо метаинформации (в виде наименования дисциплины)
	содержит указатели на функции, которые должен реализовывать модуль дисциплины обслуживания
	для работы в ядре.\cite{linuxsrc} Если не реализовать некоторые функции,
	то ядро в некоторых случаях попробует использовать функции по умолчанию, однако
	для особенно важных (к примеру, измнение конфигурации дисциплины или класса)
	сообщит пользователю, что операция не реализована.

	Поля структуры \lstinline{Qdisc_ops} представляют собой указатели на функции 
	представленными ниже сигнатурами.
	\begin{itemize}
		\item \lstinline{enqueue}\\
   		    \lstinline{int enqueue(struct sk_buff *skb, struct Qdisc *sch, struct sk_buff **to_free);} \\
			Фукнкция добавляет пакет в очередь. Если пакет был отброшен, функция
			возращает код ошибки, говорящий о том, был отброшен пришедший пакет или
			иной, чьё место занял новый.
		\item \lstinline{dequeue}\\
			\lstinline{struct sk_buff *dequeue(struct Qdisc * sch);} \\
			Функция, возвращающая пакет из очереди на отправку. Дисциплина
			может не передавать пакет при вызове этой функции по решению
			алгоритма, в таком случае вернув нулевой указатель; 
			однако то же значение алгоритм возвращает в случае, если очередь
			пуста, поэтому в таком случае дополнительно проверяется длина
			очереди.
		\item \lstinline{peek}\\
			\lstinline{struct sk_buff *peek(struct Qdisc * sch);}\\
			Функция возвращает пакет из очереди на отправку, не удаляя его из реальной очереди,
			как это делает функция \lstinline{dequeue}.
		\item \lstinline{init}\\
			  \lstinline{int init(struct Qdisc *sch, struct nlattr *arg);}\\
			  Функция инициализирует вновь созданный экземпляр дисциплины обслуживания \texttt{sch}.
			  Вторым аргументом функции является конфигурация дисциплины обслуживния, передаваемая
			  в ядро с помощью подсистемы Netlink.
		\item \lstinline{change}\\
			  \lstinline{int change(struct Qdisc *sch, struct nlattr *arg);}\\
			  Функция изменяет текущие настройки дисциплины обслуживания. 
		\item \lstinline{dump}\\
			  \lstinline{int dump(struct Qdisc *sch, struct sk_buff *skb);}\\
			  Функция отправляет по Netlink статистику дисциплины обслуживания.
	\end{itemize}

	Также структура содержит указатель на \lstinline{struct Qdisc_class_ops},
	которая описывает указатели функции исключительно для классовых дисциплин.
	Ниже приведены наиболее важные сигнатуры и их описания.\cite{linuxsrc}
	\begin{itemize}
		\item \lstinline{find}\\
			\lstinline{unsinged long find(struct Qdisc *sch, u32 classid);}\\
			Функция возвращает приведённый к \lstinline{unsinged long} адресс класса по его идентификатору (\lstinline{classid}).
		\item \lstinline{change} \\
			\lstinline{int change(struct Qdisc *sch, u32 classid, u32 parentid, struct nlattr *attr, unsinged long *arg);}\\
			Функция используется для изменения и добавления новых классов в иерархию классов. 
		\item \lstinline{tcf_block}, \lstinline{bind_tcf}, \lstinline{unbind_tcf}\\
			В данном случае, описание сигнатур не даст какой-либо значимой информации; практически
			для всех дисциплин обслуживания они идентичны. Эти функции предназначаются для работы
			системы фильтрации.
		\item \lstinline{dump_class}\\
			\lstinline{int dump_class(struct Qdisc *sch, unsinged long cl, struct sk_buff *skb, struct tcmsg *tcm);} \\
			Функция предназначается для передачи по Netlink информации о классе и дополнительной статистики, собранной
			во время функционирования класса.
	\end{itemize}

	Для классовых дисцплин, помимо описанного, реализуют классификацию пакетов, которая
	определяет класс, куда попадёт пакет. Классификация обычно выражается в функции \lstinline{classify},
	которая вызывается при добавлении пакета в очередь (функция \lstinline{enqueue}) определяет, какому классу
	 принадлежит пакет, и возвращает указатель на этот класс.
	Экземпляр структур для дисциплины обслуживания CBWFQ приведён в патче, представленном в Приложении B.

	\subsection{Алгоритм CBWFQ}

		Реализация CBWFQ требует:
		\begin{itemize}
			\item вычисление виртуального времени окончания обслуживания для каждого
				   пакета в очередях, как этого требует WFQ;
			%\item поддержку приоритетной структуры данных для хранения пакетов,
			%	  чтобы возвращать пакет с минимальным виртуальным временем
			%	  окончания обслуживания; 
			\item поддержку классов и классовых операций;
			\item фильтрацию для классификации трафика по классам.
		\end{itemize}

		

%
%		\subsubsection{Структуры хранения данных Class-Based WFQ}
%	
%			Обычно классовые дисциплины обслуживания содержат две основные структуры:
%			для описания непосредственно дисцилины и для описания класса.  
%			Структура дисциплины содержит в себе данные, которые описывают всю дисциплину:
%			это могут быть структура данных с классми (в виде списка или дерева),
%			ограничения на очереди, статистика по всей дисциплине и так далее.
%			Структура класса, соответственно, содержит непосредственно очередь и описывающие
%			класс параметры.
%
%			Для функционирования WFQ каждый класс должен содержат в себе вес.
%			В ядре Linux для того, чтобы оперировать числами с плавающей точкой, нужно
%			использовать специальные функции, которые позволяют использовать модуль
%			операций с плавающей точкой (Float Point Unit, FPU); это вычислительно
%			дорогая операция, поэтому алгоритм был изменён таким образом, чтобы
%			использовались целочисленные значения. Поэтому веса хранятся в процентах.
%
%			Определение структур представлено в Приложении B. 
%
%
%		\subsubsection{Вычисление виртуального времени}
%			\begin{algorithmic}
%				\Function{eval\_finish\_time}{Q, c, pkt}
%					\State {// Если очередь была пуста (т.е. не была активна), время сбрасывается в 0.}
%					\State s $\gets$ 0
%					\State {// Сумма весов активных очередей.}
%					\State active\_w $\gets$ ACTIVE\_WEIGHTS(Q)
%					\If {c.queue is not empty $\land$ active\_w $\neq$ 0}
%						\State {//Вычисляем виртуальное время от начала обработки.}
%        				\State {// В ядре вычисления с плавающей запятой затруднены,}
%        				\State {// из-за чего стараемся изменить вычисления так, чтобы их избегать.}
%						\State va $\gets \dfrac {(\text{pkt.arrive\_time} - \text{c.prev\_ft}) \cdot 100} {\text{active\_w}}$ 
%        				\State {// Вычисление времени начала обработки.}
%        				\State s $\gets$ MAX(c.prev\_ft, va)
%					\EndIf
%    				\State \Return s $ + \dfrac {\text{pkt.len} \cdot 100} {\text{c.weight}}$
%				\EndFunction
%			\end{algorithmic}
%
%		\subsubsection{Добавление пакета в очередь}
%
%			Алгоритм добавления пакета обычно состоит из схожих действий:
%			классификация и добавление в очередь, если есть место в очереди
%			для пакета. 
%
%			\begin{algorithmic}
%				\Function{enqueue}{Q, pkt}
%					\State {// Сначала нужно классифицировать пакет в очередь.}
%					\State {// Функция классификации определяет очередь, которой}
%					\State {// соответствует пакет, с помощью заданных фильтров}
%					\State {// и возвращает указатель на класс.}
%					\State c $\gets $ CLASSIFY(Q, pkt)
%					\State {// Отбрасываем пакет, если длина очереди достигла предела.}
%					\If {c.queue\_len < c.limit}
%						\State DROP(Q, pkt)
%					\ElsIf {\Call{q.enqueue}{pkt}}
%    					\State {// Ради экономии ресурсов, храним два виртуальных времени:}
%    					\State {// текущего и прошлого пакета; и считаем время только для}
%    					\State {// пакета, который пришёл в пустую очередь}
%    					\If {c.queue is empty}
%							\State c.prev\_ft $\gets$ c.ft
%							\State c.ft $\gets$ EVAL\_FINISH\_TIME(Q, c, pkt)
%    					\EndIf
%					\EndIf
%				\EndFunction
%			\end{algorithmic}
%
%			Специфичным добавлением CBWFQ является вычисление времени конца обработки.
%			Оно вычисляется каждый раз, когда пакет добавляется в пустую очередь.
%			Это сделано, чтобы не реализовывать дополнительные структуры для
%			хранения и тем самым тратить меньше ресурсов.
%
%		\subsubsection{Удаление пакета из очереди}
%
%			Функция удаления пакета из очереди непосредственно реализует
%			планировщик WFQ.
%
%			\begin{algorithmic}
%				\Function{dequeue}{Q}
%					\State {// Находим класс с наименьшим финальным временем.}
%					\State {// Реализуется это простым простым прохождением по списку}
%					\State {// классов.}
%					\State C $\gets$ FIND\_MIN\_FT(Q)
%					\State {// Если не найдено классов, то все очереди пусты.}
%					\If {C is null}
%						\State \Return null
%					\EndIf
%					\State pkt $\gets$ C.QUEUE.DEQUEUE(C.queue)
%					\State {// Вычисляем новое финальное время для следующего пакета}
%					\State pkt\_next $\gets$ C.QUEUE.PEEK(C.queue)
%					\State C.ft $\gets$ EVAL\_FINISH\_TIME(Q, C, pkt\_next)
%					\State \Return pkt 
%				\EndFunction
%			\end{algorithmic}


	\newpage
	\section{ИССЛЕДОВАНИЕ CLASS-BASED WFQ}

	\subsection{Описание модели}

	\subsection{Результаты исследования}


	% Заключение
	\newpage
	\phantomsection
\section*{ЗАКЛЮЧЕНИЕ}
\addcontentsline{toc}{section}{ЗАКЛЮЧЕНИЕ}


	В результате выполнения работы были достигнуты следующие цели:
	\begin{itemize}
		\item проведён сравнительный анализ дисциплин обслуживания PQ, CBQ, HTB, HFSC, FWFQ, CBWFQ;
			  дано краткое описание алгоритмов и приведены их слабые и сильные стороны;
		\item проведено исследование архитектуры подсистемы контроля качества обслуживания ядра Linux,
			  описаны механизмы работы подсистемы;
		\item реализован и протестирован модуль дисциплины обслуживания Class-Based WFQ для ядра Linux;
		\item реализован модуль для утилиты tc для взаимодействия с модулем дисциплины обслуживания. 
	\end{itemize}

	В дальнейшем работу можно развить в следующих направлениях.
	\begin{enumerate}
		\item Реализация взвешеннего алгоритм раннего обнаружения (WRED) для возможности
			  конфигурации политики отбрасывания для модуля CBWFQ;
		\item Доработка работы до дисциплины Low-Latency Queuing (LLQ), которая
			  совершенствует дисциплину CBWFQ с помощью добавления приоритетных очередей,
			  использующиеся для чувствительного к задержкам трафика.

	\end{enumerate}


	\newpage
\begingroup
\let\itshape\upshape

	% Список литературы
    \renewcommand{\refname}{СПИСОК ИСПОЛЬЗОВАННЫХ ИСТОЧНИКОВ}
    \addcontentsline{toc}{section}{СПИСОК ИСПОЛЬЗОВАННЫХ ИСТОЧНИКОВ}
    %\bibliographystyle{unsrtnat}
    \bibliography{bibl}
 
	\newpage
    \phantomsection
    \section*{СПИСОК СОКРАЩЕНИЙ И УСЛОВНЫХ ОБОЗНАЧЕНИЙ}
    \addcontentsline{toc}{section}{СПИСОК СОКРАЩЕНИЙ И УСЛОВНЫХ ОБОЗНАЧЕНИЙ}

    \begin{itemize}
		\item ДО --- Дисциплина обслуживания.
		\item ПС --- пропускная способность.
    	\item CBWFQ --- Class Based Weighted Fail Queueing.
		\item CBQ --- Class Based Queueing.
		\item ED/AD --- Early-Detection/Aggressive-Detection.
		\item HTB ---  Hierarchical Token Bucket.
		\item HFSC --- Hierarchical Fair-Service Curve.
		\item PQ --- Priority Queueing.
		\item RT/LS --- Real-Time/Link-Sharing.
		\item RR --- Round Robin.
		\item TD --- Tail-Drop.
		\item WFQ --- Weighted Fair Queueing.
		\item WRR --- Weighted Round Robin.
		\item WRED --- Weighted Random Early Detection.
    \end{itemize}

\setstretch{1.0}
	\newpage
    \phantomsection
    \section*{ПРИЛОЖЕНИЕ A}
    \addcontentsline{toc}{section}{ПРИЛОЖЕНИЕ A}
        \includecode{c}{code/pkt_sched.diff}{Патч для заголовочного файла pkt\_sched.h.}

	\newpage
    \phantomsection
    \section*{ПРИЛОЖЕНИЕ Б}
    \addcontentsline{toc}{section}{ПРИЛОЖЕНИЕ Б}
        \includecode{c}{code/q_cbwfq.c}{Модуль CBWFQ для утилиты tc.}

	\newpage
    \phantomsection
    \section*{ПРИЛОЖЕНИЕ B}
    \addcontentsline{toc}{section}{ПРИЛОЖЕНИЕ B}
        \includecode{c}{code/sch_cbwfq.c}{Модуль CBWFQ для ядра Linux.}

\end{document}
