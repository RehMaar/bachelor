\documentclass{standalone}
\usepackage{tikz}
\usepackage[utf8] {inputenc}
\usepackage[T2A]{fontenc}
\usepackage[english, russian] {babel}
% Times New Roman
\usepackage{pscyr}
\renewcommand{\rmdefault}{ftm}

\usepackage{tikz}
\usetikzlibrary{shapes,arrows, chains, scopes}

\tikzset{
	line/.style={draw, -latex'},
	every join/.style={line},
	u/.style={anchor=south},
	r/.style={anchor=west},
	fxd/.style={text width = 3em},
	it/.style={font={\small\itshape}},
	bf/.style={font={\small\bfseries}}
}
\tikzstyle{base} =
	[
		draw,
		on chain,
		on grid,
		align=center,
		minimum height=4ex,
		minimum width = 10ex,
		node distance = 6mm and 60mm,
		text badly centered
	]
\tikzstyle{coord} =
	[
		coordinate,
		on chain,
		on grid
	]
\tikzstyle{cloud} =
	[
		base,
		ellipse,
		fill = red!5,
		node distance = 3cm,
		minimum height = 2em
	]
\tikzstyle{decision} =
	[
		base,
		diamond,
		aspect=2,
		fill = green!10,
		node distance = 2cm,
		inner sep = 0pt
	]
\tikzstyle{block} =
	[
		rectangle,
		base,
		fill = blue!3,
		rounded corners,
		minimum height = 2em
	]
\tikzstyle{print_block} =
	[
		base,
		tape,
		tape bend top=none,
		fill = yellow!10
	]
\tikzstyle{io} =
	[
		base,
		trapezium,
		trapezium left angle = 70,
		trapezium right angle = 110,
		fill = blue!5
	]
\makeatletter
\pgfkeys{/pgf/.cd,
	subrtshape w/.initial=2mm,
	cycleshape w/.initial=2mm
}
\pgfdeclareshape{subrtshape}{
	\inheritsavedanchors[from=rectangle]
	\inheritanchorborder[from=rectangle]
	\inheritanchor[from=rectangle]{north}
	\inheritanchor[from=rectangle]{center}
	\inheritanchor[from=rectangle]{west}
	\inheritanchor[from=rectangle]{east}
	\inheritanchor[from=rectangle]{mid}
	\inheritanchor[from=rectangle]{base}
	\inheritanchor[from=rectangle]{south}
	\backgroundpath{
		\southwest \pgf@xa=\pgf@x \pgf@ya=\pgf@y
		\northeast \pgf@xb=\pgf@x \pgf@yb=\pgf@y
		\pgfmathsetlength\pgfutil@tempdima{\pgfkeysvalueof{/pgf/subrtshape w}}
		\def\ppd@offset{\pgfpoint{\pgfutil@tempdima}{0ex}}
		\def\ppd@offsetm{\pgfpoint{-\pgfutil@tempdima}{0ex}}
		\pgfpathmoveto{\pgfqpoint{\pgf@xa}{\pgf@ya}}
		\pgfpathlineto{\pgfqpoint{\pgf@xb}{\pgf@ya}}
		\pgfpathlineto{\pgfqpoint{\pgf@xb}{\pgf@yb}}
		\pgfpathlineto{\pgfqpoint{\pgf@xa}{\pgf@yb}}
		\pgfpathclose
		\pgfpathmoveto{\pgfpointadd{\pgfpoint{\pgf@xa}{\pgf@yb}}{\ppd@offsetm}}
		\pgfpathlineto{\pgfpointadd{\pgfpoint{\pgf@xa}{\pgf@ya}}{\ppd@offsetm}}
		\pgfpathlineto{\pgfpointadd{\pgfpoint{\pgf@xb}{\pgf@ya}}{\ppd@offset}}
		\pgfpathlineto{\pgfpointadd{\pgfpoint{\pgf@xb}{\pgf@yb}}{\ppd@offset}}
		\pgfpathclose
	}
}
\makeatother
\tikzstyle{subroutine} =
	[
		base,
		subrtshape,
		fill = green!25
	]

\begin{document}
\begin{tikzpicture} [start chain=going below]
    % Place nodes
	\draw (-7,-13.5) rectangle (5,0.5);
	\node[cloud] (start) {Начало};
	\node[block, below of = start, join, yshift=0.4cm] (init) {pkt := новый пакет\\ Q := дисциплина обслуживания};
	\node[subroutine, below of = init, join, yshift=0.4cm] (class) {cl := classify(pkt)};
	\node[decision, text width = 1.5cm, below of = class, yshift=1.6cm, join] (chklimit) {cl.len $\leq$ cl.limit};
	\node[subroutine, text width = 3cm, below of = chklimit] (enq) {ret := enqueue(pkt)};
	\node[decision, text width = 1.5cm, join, yshift=0.6cm] (enqok) {ret == 0};
	\node[block, yshift = 0.5cm] (vl) {vl := pkt.len $\cdot$ Q.rate / cl.rate};
	\node[decision, text width = 1.5cm, join,  yshift=0.6cm] (isact) {cl активен};
	\node[block, yshift=0.5cm] (update) {cl.sn := cl.sn + vl};
	\node[block, below of = update, join,yshift=.6cm] (updatepkt) {pkt.sn := cl.sn};
	\node[cloud, below of = updatepkt, yshift=2.9cm, join] (stop) {Конец};

	\node[block, below of = isact, xshift = -4cm] (updateact) {c.sn := Q.cycle + vl};
	\node[subroutine, below of = chklimit, xshift=-4cm] (drop) {drop(pkt)};

	\path[line] (chklimit) -| node[above right] {Нет} (drop);
	\path[line] (chklimit) -- node[right] {Да} (enq);

	\path[line] (enqok) -| node[below right] {Нет} (drop);
	\path[line] (enqok) -- node[right] {Да} (vl);

	\path[line] (isact) -| node[above right] {Нет} (updateact);
	\path[line] (isact) -- node[right] {Да} (update); 

	\path[line] (updateact) |- (updatepkt);
\end{tikzpicture}

\end{document}
